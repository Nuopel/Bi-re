	\documentclass[twoside,twocolumn]{report}

%\usepackage{blindtext} % Package to generate dummy text throughout this template 

\usepackage[sc]{mathpazo} 
\usepackage[T1]{fontenc} % Use 8-bit encoding that has 256 glyphs
\usepackage[utf8x]{inputenc}% accents

\usepackage[french]{babel} % Language hyphenation and typographical rules

\usepackage[hmarginratio=1:1,top=20mm,columnsep=17pt]{geometry} % Document margins
\usepackage[hang, small,labelfont=bf,up,textfont=it,up]{caption} % Custom captions under/above floats in tables or figures
\usepackage{booktabs} % Horizontal rules in tables


\usepackage{enumitem} % Customized lists
\setlist[itemize]{noitemsep} % Make itemize lists more compact
\geometry{verbose,tmargin=2cm,bmargin=2.5cm,lmargin=2cm,rmargin=2cm}
\usepackage{abstract} % Allows abstract customization
\renewcommand{\abstractnamefont}{\normalfont\bfseries} % Set the "Abstract" text to bold
\renewcommand{\abstracttextfont}{\normalfont\small\itshape} % Set the abstract itself to small italic text

\usepackage{titlesec} % Allows customization of titles
\renewcommand\thesection{\Roman{section}} % Roman numerals for the sections
\renewcommand\thesubsection{\roman{subsection}} % roman numerals for subsections
\titleformat{\section}[block]{\large\scshape\centering}{\thesection.}{1em}{} % Change the look of the section titles
\titleformat{\subsection}[block]{\large}{\thesubsection.}{1em}{} % Change the look of the section titles

\usepackage{fancyhdr} % Headers and footers
\pagestyle{fancy} % All pages have headers and footers
\fancyhead{} % Blank out the default header
\fancyfoot{} % Blank out the default footer
\fancyhead[C]{Porous materials   $\bullet$ January 2016 } % Custom header text
\fancyfoot[RO,LE]{\thepage} % Custom footer text

\usepackage{titling} % Customizing the title section

\usepackage{hyperref} % For hyperlinks in the PDF
\usepackage{amsmath} % Customizing the title section
\usepackage{graphicx,epstopdf}
\usepackage{mathtools}
\usepackage[colorinlistoftodos,prependcaption,textsize=tiny]{todonotes}

%% biber use
%\usepackage[autostyle]{csquotes}
%\usepackage[backend=biber,style=authoryear-icomp,sortlocale=de_DE,natbib=true,url=false, doi=true,eprint=false]{biblatex}
%\addbibresource{Thesis.bib}
%----------------------------------------------------------------------------------------
%	TITLE SECTION
%----------------------------------------------------------------------------------------

\setlength{\droptitle}{-4\baselineskip} % Move the title up

\pretitle{\begin{center}\Huge\bfseries} % Article title formatting
\posttitle{\end{center}} % Article title closing formatting
\title{Guide Bière } % Article title
\author{%
\textsc{Samuel Dupont}\\ %\thanks{A thank you or further information} \\[1ex] 
%\normalsize Université du Maine \\ % Your institution
\normalsize \href{mailto:Samuel.dupont.etu@univ-lemans.fr}{Samuel.dupont.etu@univ-lemans.fr } 
}

\date{Février 10, 2017 \\ Last update: \today}

\begin{document}
% Print the title
\maketitle

\chapter*{Guide de fabrication de bière étapes par étape}
	\section{Introduction}
		Ce petit guide à pour but de dicter chaque étapes à faire pour fabriquer la bière, il n'a donc pas pour but d'expliquer les processus en cours mais bien juste d'être une ligne à suivre durant les opérations.
		
	\section{Outillage}
			Afin de réaliser les opérations un minimum d'outils est requis: 
			\begin{itemize}
				\item Deux cuves de fermentation.
				\item Un robinet à monter sur une cuve.
				\item Un thermomètre.
				\item Un kit Houblon malt.
				\item Une casserole de 10 L.
				\item Une système de chauffe.
				\item Du sucre.
				\item Une cuillère.
				\item Des bouteilles.	
				\item Du savon anti-bactérien et des éponges.
				\item Une  grande passoire.	 
		\end{itemize}
		
	\section{Préparation}
		 Avant de se lancer dans la cuisson de la popote, il faut bien nettoyer tout le matériel qui va être en contact avec la bière (Cuve, robinet, barboteur, thermomètre).\\
		 C'est une étape critique, l'infection de la bière par des bactéries est une des première cause de bière brassé raté.
	
	\section{Empâtage}
		L'empâtage consiste à infuser le malt dans de l'eau, il transforme l'amidon en sucres simples.\\
		Cela s'effectue entre 66°C et 68°C.
		\subsection{Cuisson de l'eau}
			Faites chauffer la casserole d'eau (du litrage désiré par rapport à la quantité de bière désiré) à 71°C.\\
		\subsection{Ajout du Malt}
			Ajouter le malt dans l'eau, la température descendra naturellement autour de 67°.
		\subsection{Infusion}
			Laisser le malt infuser durant 85 min autour de 66°C, 68°C tout en remuant régulièrement.\\ \\
		\subsection{ Mash Out}
			 Augmentez la température durant les 10 dernières minutes de l’empâtage pour atteindre environ 75°C. Cela  permet la destruction des
			enzymes et la stabilisation de votre moût  (jus  sucré issu  de l’empâtage).
		\subsection{Transvasement }
			Renverser le liquide dans une passoire au dessus de la seconde cuve.\\
		\subsection{Rinçage des drêches}
			Verser le complément d'eau prévus dans la recette préalablement chauffé à 68° sur le malt dans la passoire. L'eau ajouté finissant avec la première partie de l'eau dans la seconde cuve.\\
			Cela permet de récupérer le sucre restant dans le malt. Le liquide obtenu est appelé le mout.
			
		\section{Ébullition et houblonnage}
			Cette étape à pour but de stériliser le mout et de stopper l'action des enzymes du malt.\\ L'ajout de Houblon donne l'amertume à la bière.
			\subsection{Ébullition}
				Porter le liquide à ébullition.
			\subsection{Ajout du houblon Amer }
				Ajouter le houblon amer dès que le mout est à ébullition.
			\subsection{Remuage}
				Laisser bouillir pendant 50 min en  remuant régulièrement.\\ \\
				
				Astuce : 
				\begin{itemize}
					\item Fermer la casserole.\\
					\item Si le mout déborde ça colle partout!\\
					\item Enlever l'écume si besoin.
				\end{itemize}
			\subsection{Ajout du houblon aromatique}
				Laisser bouillir pendant 10 min en  remuant régulièrement.
			\subsection{Filtrage}
				Filtrer le liquide en versant dans une cuve en pensant à bien fermer le robinet.\\
				
		\section{Refroidissement et Préparation à la fermentation}
			\subsection{Refroidissement}
				Afin de refroidir efficacement plonger la cuve dans un évier remplie à moitié d'eau froide.
			\subsection{Ajout des levure}
				Préparer la levure en la réhydratant, soit en récupérant 10 cL de moût, soit en faisant bouillir puis refroidir 10 cL d’eau à 20-25°C.\\
				Saupoudrez ensuite les levures à la surface	du liquide. Lorsque celles-ci se sont accumulées au fond, la levure est prête.\\ \\
												
				Une fois que le liquide à atteint 20-25 degrés, ajouter la levure.\\
			
			\subsection{Préparation du barboteur}
				Remplir à moitié le barboteur avec de l’eau, puis le placer sur le bouchon
				de la cuve.
		 
		 \section{Fermentation}
		 Cette étape permet 
			 \subsection{Stockage}
				  Placez votre cuve dans un lieu sec, à la température la
				 plus constante possible aux environs de 18-20°C à l’abri de
				 la lumière  du soleil.\\
				 Une température de 25°C donne un goût de levure à la bière et une température de 30°C tue les levures.\\
			 \subsection{Garde}
				 Au bout de 15 Jours, quand les signes de fermentation disparaissent (plus d’activité dans le barboteur) la bière doit subir une phase de maturation dite de "garde".\\
				 Il faut pour cela transporter le récipient dans un lieu plus froid ( 6 à 10°C, en tout cas pas en dessous de -2°C pour ne pas tuer les levures ) et le laisser de une à quelques semaines.
			 \subsection{Soutirage}
				 Transvaser le liquide via le robinet de soutirage dans un autre récipient en laissant le dépôt au fond de la cuve.\\
				 Nettoyer la cuve.\\
				 Remettre le liquide dans la cuve.\\
				 
	 \section{Affinage}
			 Pour rendre la bière mousseuse il va falloir la faire re- fermenter en bouteille : il suffit pour cela	 de la sucrer, pour relancer une fermentation avec les levures encore présentes, à l’embouteillage.
			 \subsection{Ajout du sucre}
				 Quantité de sucre à l’embouteillage : 5 à 6g par litre.\\	 
				 Faites bouillir une  masse d’eau égale au sucre, ajoutez le sucre. \\
				 Mélanger et verser froid dans la cuve.\\
			 \subsection{Embouteillage}
				 Nettoyer chaque bouteilles et bouchons.\\
				 Remplissez chaque bouteille avec le robinet  jusqu’à  mi-goulot .\\
				 Refermez les  bouteilles.\\
				 
				 Placez les bouteilles verticalement à la température d’environ 18-20°C à l'abri du soleil pendant une semaine puis une fois la refermentation terminée placez vos bouteilles au frais (environ 5°C) pour une
				 nouvelle garde. Cette maturation doit durer autour de 3-4 semaines.
				 
				 
				 
				 
				 
				 
				 
				 
				 
				 
				
				
				
				
			

			
				
						
		
 

	
\end{document}