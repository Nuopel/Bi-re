	\documentclass[twoside,twocolumn]{report}

%\usepackage{blindtext} % Package to generate dummy text throughout this template 

\usepackage[sc]{mathpazo} 
\usepackage[T1]{fontenc} % Use 8-bit encoding that has 256 glyphs
\usepackage[utf8x]{inputenc}% accents

\usepackage[french]{babel} % Language hyphenation and typographical rules

\usepackage[section]{placeins}% allow graph in section
\usepackage[hmarginratio=1:1,top=20mm,columnsep=17pt]{geometry} % Document margins
\usepackage[hang, small,labelfont=bf,up,textfont=it,up]{caption} % Custom captions under/above floats in tables or figures
\usepackage{booktabs} % Horizontal rules in tables


\usepackage{enumitem} % Customized lists
\setlist[itemize]{noitemsep} % Make itemize lists more compact
\geometry{verbose,tmargin=2cm,bmargin=2.5cm,lmargin=2cm,rmargin=2cm}
\usepackage{abstract} % Allows abstract customization
\renewcommand{\abstractnamefont}{\normalfont\bfseries} % Set the "Abstract" text to bold
\renewcommand{\abstracttextfont}{\normalfont\small\itshape} % Set the abstract itself to small italic text

\usepackage{titlesec} % Allows customization of titles
\renewcommand\thesection{\Roman{section}} % Roman numerals for the sections
\renewcommand\thesubsection{\roman{subsection}} % roman numerals for subsections
\titleformat{\section}[block]{\large\scshape\centering}{\thesection.}{1em}{} % Change the look of the section titles
\titleformat{\subsection}[block]{\large}{\thesubsection.}{1em}{} % Change the look of the section titles

\usepackage{fancyhdr} % Headers and footers
\pagestyle{fancy} % All pages have headers and footers
\fancyhead{} % Blank out the default header
\fancyfoot{} % Blank out the default footer
\fancyhead[C]{ Guide kit bière   $\bullet$ Printemps 2017 } % Custom header text
\fancyfoot[RO,LE]{\thepage} % Custom footer text

\usepackage{titling} % Customizing the title section

\usepackage{hyperref} % For hyperlinks in the PDF
\usepackage{amsmath} % Customizing the title section
\usepackage{graphicx,epstopdf}
\usepackage{mathtools}
\usepackage[colorinlistoftodos,prependcaption,textsize=tiny]{todonotes}

%% biber use
%\usepackage[autostyle]{csquotes}
%\usepackage[backend=biber,style=authoryear-icomp,sortlocale=de_DE,natbib=true,url=false, doi=true,eprint=false]{biblatex}
%\addbibresource{Thesis.bib}
%----------------------------------------------------------------------------------------
%	TITLE SECTION
%----------------------------------------------------------------------------------------

\setlength{\droptitle}{-4\baselineskip} % Move the title up

\pretitle{\begin{center}\Huge\bfseries} % Article title formatting
\posttitle{\end{center}} % Article title closing formatting
\title{Guide Bière } % Article title
\author{%
\textsc{Samuel Dupont}\\ %\thanks{A thank you or further information} \\[1ex] 
%\normalsize Université du Maine \\ % Your institution
\normalsize \href{mailto:Samuel.dupont.etu@univ-lemans.fr}{Samuel.dupont.etu@univ-lemans.fr } 
}

\date{Février 10, 2017 \\ Last update: \today}

\begin{document}
% Print the title
\maketitle

	Guide de fabrication de bière étapes par étape.
	\chapter*{Fabrication Bière 1 : Kit Tarwebier blanche}
		\section{Introduction}
			J'ai acheté en même temps que le matériel un Kit à malt-houblon pour faire une première cuvée de bière!\\
			Je m'attendais à devoir faire l'empâtage (La cuisson du malt et du houblon) mais c'était du prémâché. Du coup c'étais un peu décevant, j'ai juste eu à réchauffer une boite de conserve et verser le contenu dans une cuve avec de l'eau... La suite en images!
		\section{Nettoyage}
			Une bonne bière commence par un bon nettoyage avec une solution hydrogéné en utilisant du chemipro.
			\begin{figure}[h!]
				\centering
				\includegraphics[width=200px]{./photos/init/nettoyage2b.JPG}
				\caption{Nettoyage du matériel.}
				\label{netoyage}
			\end{figure}
			\begin{figure}[h!]
				\centering
				\includegraphics[width=200px]{./photos/init/chemichemiprob.JPG}
				\caption{Chemichemipro!}
				\label{chemi}
			\end{figure}
		\section{L'empâtage, infusion du malt et houblonnage} 
			Dire que j'ai fais l'empâtage,l'infusion et le houblonnage est un bien grand titre pour pas grand chose, j'ai réchauffé au bain marie une boite de conserve. \\
			 \begin{figure}[h!]
			 	\centering
			 	\includegraphics[width=200px]{./photos/init/conserve1b.JPG}
			 	\caption{Boite de conserve au bain marie.}
			 	\label{boite}
			 \end{figure}
			Voici le coté ustensile, après en j'ai fait chauffé l'eau du robinet pour stériliser et faire évaporer le chlore mais ce n'était pas particulièrement utile :
			\begin{figure}[h!]
				\centering
				\includegraphics[width=200px]{./photos/init/Matosb.JPG}
				\caption{Le matériel.}
				\label{matos}
			\end{figure}
			J'ajoute la boite de conserve avec un sirop de 750g de sucre, dans la cuve avec 11 L d'eau:
			\begin{figure}[h!]
				\centering
				\includegraphics[width=200px]{./photos/init/cuve3b.JPG}
				\caption{Ajout de la boite de conserve.}
				\label{cuve}
			\end{figure}
			\begin{figure}[h!]
			\centering
			\includegraphics[width=200px]{./photos/init/cuveremplieb.JPG}
			\caption{Cuve remplie.}
			\label{cuver}
			\end{figure}
			\section{Ajout des levures}
			Les levures étant sèches, je les réhydrates avant de les mettre avec de l'eau à 25°, comme dans la photo \ref{levure}.\\ \\
			\begin{figure}[h!]
				\centering
				\includegraphics[width=200px]{./photos/init/levureb.JPG}
				\caption{Ajout des levures.}
				\label{levure}
			\end{figure}
			Afin d'éviter de tuer les levures, j'attends que la température de la cuve baisse à 25°, en la mettant dehors \ref{refroid}.
			\begin{figure}[h!]
				\centering
				\includegraphics[width=200px]{./photos/init/refroidissementb.JPG}
				\caption{Refroidissement de la cuve avant l'ajout des levures.}
				\label{refroid}
			\end{figure}
		
		\section{Fermentation 1}
			Je met la cuve à température ambiante 20°-23° et je la cache de la lumière avec une couverture  \ref{cuvec}!\\
			\begin{figure}[h!]
				\centering
				\includegraphics[width=200px]{./photos/init/cuveCacheb.JPG}
				\caption{Cuve caché !}
				\label{cuvec}
			\end{figure}
			Finalement on voit sur la photo que le barboteur bloblote ! Donc tout va bien !
			 \begin{figure}[h!]
			 	\centering
			 	\includegraphics[width=200px]{./photos/init/blob2b.JPG}
			 	\caption{Le barboteur.}
			 	\label{blob}
			 \end{figure}
			A dans deux semaines!
				 
		\section{Soutirage et fermentation 2}		 
		Après deux semaines les levures ont fini leur travail (et sont mortes), il faut donc soutirer le liquide afin de les enlever.\\
		J'utilise un deuxième seau préalablement lavé (avec du chemichemipro!!), le liquide est vidé par le robinet prévus à cet effet \ref{soutirage}. Et déjà une super odeur de Bière! \\
		
			\begin{figure}[h!]
				\centering
				\includegraphics[width=200px]{./photos/fermentation2/Soutirage1.jpg}
				\caption{Soutirage dans la seconde cuve.}
				\label{soutirage}
			\end{figure}	
			Comme on peut le voir sur la photo \ref{levu}, il ne reste dans le fond qu'un dépôt jaunes sable avec un peu de liquide restant, dues aux levures.
			\begin{figure}[h!]
				\centering
				\includegraphics[width=200px]{./photos/fermentation2/fond4levure.jpg}
				\caption{Fond de la cuve, les levures mortes.}
				\label{levu}
			\end{figure}
			Pour le plaisir une photo de la cuve pleine \ref{full}.\\ \\
			\begin{figure}[h!]
				\centering
				\includegraphics[width=200px]{./photos/fermentation2/fullbucket.jpg}
				\caption{Second seau, plein.}
				\label{full}
			\end{figure}
			Le premier seau est nettoyé \ref{clean}, les dépôts de levures sur le bord accrochent beaucoup, il faut y passer du temps...\\ \\
			\begin{figure}[h!]
				\centering
				\includegraphics[width=200px]{./photos/fermentation2/clean.jpg}
				\caption{Nettoyage.}
				\label{clean}
			\end{figure} 
			Le contenu est re-transvasé dans le premier seau (parce qu'il a un robinet utile pour remplir les bouteilles ). Et enfin il est remis à fermenté à 17-20° \ref{hidden}!\\ \\
			C'est reparti pour 2-3 semaines! Au passage la bière à goût de bière amer, sans sucre pour le moment, bref c'est pas bon (j'ai pas résisté au fait de goûter :p).
			\begin{figure}[h!]
				\centering
				\includegraphics[width=200px]{./photos/fermentation2/hidenbuck.jpg}
				\caption{Re fermentation.}
				\label{hidden}
			\end{figure} 
				 
		\section{Embouteillage }
		Et hop ! Après 4 semaines la seconde fermentation est fini! \\
		Avant tout je nettoie le matériel que je vais utiliser: une spatule, le seau pour récupérer le liquide soutiré et bien sur les bouteilles misent de coté pour l'encapsulage.\\
		\\
		Une fois fait, je commence à soutirer le liquide comme sur la photo \ref{secondsou}. On peu voir que le liquide est quand même bien opaque (ce qui est bizarre pour une bière blanche), mais passons, ça sent super bon.\\
		Il reste un dépôt comme la dernière fois photo \ref{levu} mais cette fois si d'à peine 1mm.\\ \\
		Je nettoie la cuve puisque c'est la seule qui à un robinet et je remets 
		tout dans celle si pour préparer pouvoir utiliser le robinet avec les bouteilles.(Il faudra que je pense à percer le $\text{2}^{eème}$ seau et y mettre un robinet).
		
		\begin{figure}[h!]
			\centering
			\includegraphics[width=200px]{./photos/Embout/SoutirageOk.jpg}
			\caption{Second soutirage après 4 semaines.}
			\label{secondsou}
		\end{figure}
		J'ajoute le sucre nécessaire à la garde.\\
		La mise en bouteille est effectué sur la photo \ref{Bout}.
		\begin{figure}[h!]
			\centering
			\includegraphics[width=200px]{./photos/Embout/MiseEnBouteille.jpg}
			\caption{Mise en bouteille.}
			\label{Bout}
		\end{figure}
		A l'aide des capsule photo \ref{cap} j'encapsule les bouteilles voir \ref{Encap}.
		\begin{figure}[h!]
			\centering
			\includegraphics[width=200px]{./photos/Embout/Capsule.jpg}
			\caption{Capsules.}
			\label{cap}
		\end{figure}
	 	\begin{figure}[h!]
	 		\centering
	 		\includegraphics[width=200px]{./photos/Embout/Encapsulage2.jpg}
	 		\caption{Encapsulages.}
	 		\label{Encap}
	 	\end{figure}
	 	Et paf j'ai 18 bouteilles de prêtes! J'ai plus qu'à attendre 2 semaines pour laisser les bulles se faire et quelques une supplémentaire pour laisser affiner avant dégustation! 
	 	\begin{figure}[h!]
	 		\centering
	 		\includegraphics[width=200px]{./photos/Embout/Bouteilles.jpg}
	 		\caption{Résultats des bouteilles.}
	 		\label{Embouteillage}
	 	\end{figure}
	 	
		 	
				 
	
		
				
				
				
			

			
				
						
		
 

	
\end{document}