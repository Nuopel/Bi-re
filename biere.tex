\documentclass{report}
%\usepackage{blindtext} % Package to generate dummy text throughout this template 

\usepackage[sc]{mathpazo} 
\usepackage[T1]{fontenc} % Use 8-bit encoding that has 256 glyphs
\usepackage[utf8]{inputenc}% accents

\usepackage[french]{babel} % Language hyphenation and typographical rules

\usepackage[a4paper]{geometry}
\geometry{verbose,tmargin=3cm,bmargin=4cm,lmargin=3cm,rmargin=3cm}
\setcounter{secnumdepth}{3}
\setcounter{tocdepth}{3}

\usepackage{textcomp}
\usepackage[unicode=true,bookmarks=true,bookmarksnumbered=false,bookmarksopen=false,
 breaklinks=false,pdfborder={0 0 1},backref=false,colorlinks=false]
 {hyperref}
\usepackage{breakurl}
\usepackage{array}
\usepackage{multirow}
\usepackage{makecell}
\usepackage{caption}
\usepackage{graphicx}
\usepackage[section]{placeins}

\usepackage{listings}
\usepackage{xcolor}    
\makeatletter
%useful packages
%\usepackage[siunitx]{circuitikz}
%\usepackage{tikz}
%\usepackage{pgfplots}
%\usepackage{pgfplotstable}
%\usepackage{booktabs}
%\usepackage{placeins}

\makeatother

\begin{document}
\pagestyle{empty} %No headings for the first pages.
\newcommand{\HRule}{\rule{\linewidth}{1mm}}
\begin{center}
\vspace*{\stretch{0.3}}
\end{center}


\begin{center}
  \large
  \textbf{Fabriquer de la bière}\\
  \vspace*{\stretch{0.2}}
  \huge

  \vspace*{\stretch{0.8}}
  \small

\end{center}

\chapter{Introduction}
\paragraph{"Parce que la bière, c'est bon", Ki}

\chapter{Matériel nécessaire}
\section{Ingrédients}
Les différents composants de la bière sont :
\begin{itemize}
\item    l'eau
\item    le malt (orge germé puis chauffé)
\item    le houblon
\item    les levures
\end{itemize}
 
\subsection*{L'eau}

L'eau de source est préférable ; elle doit avoir de préférence un pH de 5 à 5,5 : les enzymes qui vont dégrader l'amidon travaillant mieux dans un milieu acide.
 


\section{Outillage}
La propreté du matériel est primordiale pour la qualité de la bière. En aucun cas il ne faut utiliser de l'aluminium qui risque de réagir avec la bière. Les cuves doivent ètre en inox (matériel stable et lavage très aisé) ou en plastique qui ne réagisse pas avec les mélanges.
Voici une liste d'un équipement minimum pour pouvoir brasser :
\begin{itemize}
\item    une cuve ou grand fait-tout pour l'empatage et le brassage.
\item    une cuve ou grand fait-tout pour chauffer l'eau de lavage de la drèche.
\item     une cuve ou grand fait-tout pour l'ébullition.
\item    une passoire conséquente pour le lavage de la drèche (une bassine en plastique. alimentaire perforée avec trous de 2mm fera l'affaire).
\item    un bruleur puissant (gain de temps) ou une plaque de cuisson.
\item   un moulin à céréales si le malt n'est pas broyé.
\item   un récipient pour la fermentation plus barboteur.
\end{itemize}

\chapter{Les étapes}
\section{Le maltage}

Le principe du maltage consiste à produire les enzymes nécessaires à la saccharification de l'amidon, et donc, à la fabrication de l'alcool lors de la fermentation.\\
Ceci se passe en quatre étapes :
\begin{itemize}
  \item le trempage
  \item la germination
   \item le touraillage (séchage : c'est la température atteinte en fin de touraillage qui va déterminer la couleur du malt.)
   \item le dégermage
\end{itemize} 

Le malt est en général acheté
\paragraph{Ingrédients : orge + eau}

\section{Le brassage}

Cette étape, appelée aussi saccharification, consiste à transformer les sucres complexes (amidon) contenus dans le grain en sucres simples fermentescibles (glucose, maltose et dextrose) et non fermentescibles (dextrines), grace à l'action des enzymes du malt, activées par chauffage du malt dans de l'eau.\\
On filtre et rince à l'eau chaude ce mélange (maische) pour obtenir le mout. 

\paragraph{Ingrédients : eau + malt}


\section{L'ébullition - le houblonnage}

Cette étape a pour objectif la destruction des enzymes, la stérilisation et la stabilisation du mout.
C'est à ce moment que le houblon est ajouté : le houblon possède des vertus conservatrices, donne de l'amertume et produit des arômes.


\paragraph{Ingrédients : mout + houblon}

\section{La fermentation}
La fermentation est l'étape à laquelle on ajoute des levures afin de produire l'alcool. Pour produire de la bière il existe plusieurs type de fermentation :
\begin{itemize}
\item  Basse : elle se déroule entre 5 °C et 14 °C.
\item    Haute : elle se déroule entre 15 °C et 20 °C.
\item   Spontanée : sans ajout de levures cultivées
\end{itemize}
   
    

La fermentation est généralement suivie par une clarification par décantation (garde).

\paragraph{Ingrédients : bière + levures}






\chapter{Au travail}

Le travail se fait en trois grandes étapes : l'empatage, le brassage et le houblonnage.

\section{L'empatage}
Il faut tout d'abord concasser assez grossièrement les grains : un moulin à céréales à réglage variable est l'idéal.

\subsection*{Quantité de malt d'empatage :}

Le malt blond peut ètre utilisé seul car il est le seul à contenir encore les enzymes pour transformer l'amidon des grains en sucre. Les malts plus foncés (parce que plus chauffés) apportent chacun des gouts et des arômes spécifiques. Ils sont utilisés en faibles quantités.\\

Quantité de malt blond nécessaire : 
\begin{equation}
M=\frac{B*D}{20}
\end{equation}
ou
\begin{itemize}
\item M = quantité de malt en kilo
\item B = quantité de bière désirée
\item  D = degré de bière désiré
\end{itemize}


Une fois le malt concassé, chauffez une quantité Q d'eau jusqu'à 50°, versez le malt concassé et brassez cette maische en maintenant la température pendant 20 à 30 minutes. 

Quantité d'eau pour l'empatage : $Q = M * 4$

\section{Brassage}
Montez ensuite la température jusqu'à 62~63°C et maintenez cette température pendant 30mn tout en brassant (à cette température l'amidon est transformé en maltose, un sucre fermentescible qui sera transformé en alcool pendant la fermentation).\\
Augmentez encore la température pour arriver à 66~68°C et maintenez cette température pendant 30mn tout en brassant (à cette température l'amidon est transformé en dextrose, sucre non fermentescible qui donnera la rondeur en bouche).\\
Ensuite, augmentez encore la température pour chauffer la maische jusqu'à 75~77°C et maintenir cette température 10mn.


\subsection*{ Filtrage de la maische et lavage de la drèche }

Une des possibilités pour filtrer la maische est de la verser dans une bassine perforée.\\
Il faut ensuite laver cette drèche ainsi obtenue, car il reste encore un peu de sucre, en passant de l'eau chaude à 80°C. \\
Quantité d'eau pour passer la drèche :$ E = (B * 1.25) - Q * 0,7$

\section{HOUBLONNAGE }

Le mout doit maintenant ètre amené à ébullition.\\
En début d'ébullition incorporez le houblon amer. Laissez bouillir 1 heure 30 à deux heures afin d'extraire les résines amères puis 5 à 10 minutes avant la fin de cette opération ajoutez le houblon aromatique. Videz la cuve en filtrant le houblon.\\

Quantité de houblon pour 20 litres de bière : 50 à 100 g de houblon amer et environ 20 g de houblon aromatique.

\section{FERMENTATION }

Faites refroidir le liquide. Quand il atteint une température de 20~25°C incorporez la levure (un sachet pour 20 litres de bière).
Fermez votre récipient de fermentation et installez un barboteur pour permettre l'évacuation du CO2.
La fermentation doit se faire à environ 20°C, une température de 25°C donne un gout de levure à la bière et une température de 30°C tue les levures !
Quand les signes de fermentation disparaissent (plus d'activité dans le barboteur) la bière doit subir une phase de maturation dite de "garde". Il faut pour cela transporter votre récipient dans un lieu plus froid (~ 6 à 10°C, en tout cas pas en dessous de -2°C pour ne pas tuer les levures ) et le laisser de 1 à quelques semaines.

\section{EMBOUTEILLAGE}

Pour rendre la bière mousseuse il va falloir la faire refermenter en bouteille : il suffit pour cela de la sucrer, pour relancer une fermentation avec les levures encore présentes, à l'embouteillage.\\

Quantité de sucre à l'embouteillage : 4 à 6g par litre.

Le plus simple est de sucrer l'ensemble et de mettre en bouteille après.

\paragraph{--> ATTENTION TROP DE SUCRE ET LA BOUTEILLE SE VIDE SEULE LORS DE SON OUVERTURE à CAUSE DE LA SURPRESSION !}

\section{REFERMENTATION ET MATURATION }

Placez les bouteilles verticalement à la température d'environ 20°C pendant une semaine puis une fois la refermentation terminée placez vos bouteilles au frais (environ 5°C) pour une nouvelle garde. Cette maturation doit durer 2 mois.



\chapter*{Bibliographie}
http://christian.seon.free.fr/fabrication-biere/biere.html

\end{document}